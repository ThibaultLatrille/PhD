\documentclass{article}
\usepackage{xcolor}
\definecolor{BLUELINK}{HTML}{0645AD}
\definecolor{DARKBLUELINK}{HTML}{0B0080}
\definecolor{LIGHTBLUELINK}{HTML}{3366BB}
\definecolor{PURPLELINK}{HTML}{663366}
\PassOptionsToPackage{hyphens}{url}
\usepackage[colorlinks=false ]{hyperref}
% for linking between references, figures, TOC, etc in the pdf document
\hypersetup{colorlinks,
linkcolor=DARKBLUELINK,
anchorcolor=DARKBLUELINK,
citecolor=DARKBLUELINK,
filecolor=DARKBLUELINK,
menucolor=DARKBLUELINK,
urlcolor=BLUELINK
} % Color citation links in purple
\PassOptionsToPackage{unicode}{hyperref}
\PassOptionsToPackage{naturalnames}{hyperref}

\usepackage[margin=60pt]{geometry}
\usepackage{amssymb,amsfonts,amsmath,amsthm,mathtools}
\usepackage{lmodern}
\usepackage{bm,bbold}
\usepackage{verbatim}
\usepackage{float}
\usepackage{listings, enumerate, enumitem}
\usepackage[export]{adjustbox}
\usepackage{tabu}
\tabulinesep=0.6mm
\newcommand\cellwidth{\TX@col@width}
\usepackage{hhline}
\setlength{\arrayrulewidth}{1.2pt}
\usepackage{multicol,multirow,array}
\usepackage{etoolbox}
\AtBeginEnvironment{tabu}{\footnotesize}
\usepackage{booktabs}

\usepackage{graphicx}
\graphicspath{{artworks/}}
\pdfstringdefDisableCommands{%
\renewcommand*{\bm}[1]{#1}%
% any other necessary redefinitions
}
\usepackage{xfrac, nicefrac}
\usepackage{natbib}
\pdfinclusioncopyfonts=1
\definecolor{RED}{HTML}{EB6231}
\definecolor{YELLOW}{HTML}{E29D26}
\definecolor{BLUE}{HTML}{5D80B4}
\definecolor{LIGHTGREEN}{HTML}{6ABD9B}
\definecolor{GREEN}{HTML}{8FB03E}
\definecolor{PURPLE}{HTML}{BE1E2D}
\definecolor{BROWN}{HTML}{A97C50}
\definecolor{PINK}{HTML}{DA1C5C}

\newcommand{\specialcell}[2][c]{%
	\begin{tabular}[#1]{@{}c@{}}#2\end{tabular}}

\DeclareMathOperator{\E}{\mathbb{E}}
\newcommand{\der}{\mathrm{d}}
\newcommand{\angstrom}{\text{\normalfont\AA}}
\newcommand{\e}{\mathrm{e}}
\newcommand{\dnds}{d_N / d_S}
\newcommand{\indice}{l}
\newcommand{\indiceexp}{^{(\indice)}}
% Time, effective population size and mutation rate.
\newcommand{\Ne}{N_{\textrm{e}}}
% \acrshort{DNA}
\newcommand{\SetNuc}{\Omega_{\mathrm{N}}}
\newcommand{\SetWeak}{\Omega_{\mathrm{W}}}
\newcommand{\SetStrong}{\Omega_{\mathrm{S}}}
\newcommand{\mutmatrix}{R}
\newcommand{\Mutmatrix}{\bm{\mutmatrix}}
\newcommand{\exchan}{\rho}
\newcommand{\Exchan}{\bm{\exchan}}
\newcommand{\mutequi}{\sigma}
\newcommand{\Mutequi}{\bm{\mutequi}}
% Codons
\newcommand{\SetCodon}{\Omega_{\mathrm{C}}}
\newcommand{\ci}{{i}}
\newcommand{\cj}{{j}}
\newcommand{\iSetCodon}{1 \leq \ci \leq 61}
\newcommand{\jSetCodon}{1 \leq \cj \leq 61}
\newcommand{\ijSetCodon}{1 \leq \ci, \cj \leq 61}
\newcommand{\itoj}{\ci, \cj}
\newcommand{\jtoi}{\cj, \ci}
\newcommand{\nucitoj}{\mathcal{M}(\itoj)}
\newcommand{\submatrix}{Q}
\newcommand{\Submatrix}{\bm{\submatrix}}
\newcommand{\subequi}{\pi}
\newcommand{\Subequi}{\bm{\subequi}}
\newcommand{\probmatrix}{P}
\newcommand{\Probmatrix}{\bm{\probmatrix}}
% Amino-acids
\newcommand{\aminoacid}{\text{A}}
\newcommand{\aSetAa}{1 \leq \aminoacid \leq 20}
\newcommand{\SetAa}{\Omega_{\mathrm{A}}}
\newcommand{\Neighbor}{\mathcal{V}}
\newcommand{\NonSyn}{\mathcal{N}}
\newcommand{\Syn}{\mathcal{S}}
\newcommand{\Nx}{\Neighbor_x}
\newcommand{\NxAB}{\Neighbor_x^{\mathrm{A} \rightarrow \mathrm{B}}}
\newcommand{\NyBA}{\Neighbor_x^{\mathrm{B} \rightarrow \mathrm{A}}}
\newcommand{\NxWS}{\Neighbor_x^{\mathrm{W} \rightarrow \mathrm{S}}}
\newcommand{\NxSS}{\Neighbor_x^{\mathrm{S} \rightarrow \mathrm{S}}}
\newcommand{\NxSW}{\Neighbor_x^{\mathrm{S} \rightarrow \mathrm{W}}}
\newcommand{\NxWW}{\Neighbor_x^{\mathrm{W} \rightarrow \mathrm{W}}}
\newcommand{\NyWS}{\Neighbor_y^{\mathrm{W} \rightarrow \mathrm{S}}}
\newcommand{\NySS}{\Neighbor_y^{\mathrm{S} \rightarrow \mathrm{S}}}
\newcommand{\NySW}{\Neighbor_y^{\mathrm{S} \rightarrow \mathrm{W}}}
\newcommand{\NyWW}{\Neighbor_y^{\mathrm{W} \rightarrow \mathrm{W}}}
\newcommand{\NxNonSyn}{\NonSyn_x}
\newcommand{\NyNonSyn}{\NonSyn_y}
\newcommand{\NxSyn}{\Syn_x}
\newcommand{\NySyn}{\Syn_y}
\newcommand{\aai}{\mathcal{A}(\ci)}
\newcommand{\aaj}{\mathcal{A}(\cj)}
\newcommand{\Ni}{\mathcal{N}_{\mathrm{eighbors}}\left(\ci\right)}
\newcommand{\NiNonSyn}{\mathcal{N}_{\mathrm{onSyn}}\left(\ci\right)}
\newcommand{\NiSyn}{\mathcal{S}_{\mathrm{yn}}\left(\ci\right)}
\newcommand{\fit}{f}
\newcommand{\Fit}{\bm{\fit}}
\newcommand{\fiti}{\fit_{\aai}}
\newcommand{\fitj}{\fit_{\aaj}}
\newcommand{\Fiti}{F_{\aai}}
\newcommand{\Fitj}{F_{\aaj}}
\newcommand{\scaledfit}{F}
\newcommand{\ScaledFit}{\bm{\scaledfit}}
\newcommand{\scaledfiti}{\scaledfit_{\aai}}
\newcommand{\scaledfitj}{\scaledfit_{\aaj}}
\newcommand{\selcoef}{{\delta_{\fit}}}
\newcommand{\scaledselcoef}{{\Delta \scaledfit}}
% Categories
\newcommand{\Seqitoj}{\ci, \cj}
\newcommand{\setNeighbors}{\mathcal{M}\left(\ci\right)}
\newcommand{\setNonSynNeighbors}{\mathcal{N}\left(\ci\right)}
\newcommand{\setSynNeighbors}{\mathcal{S}\left(\ci\right)}


\begin{document}
\part*{Supplementary materials}
\tableofcontents

\section{Simulated alignments}
\label{sec:simulated-alignments}

\subsection{Primate phylogeny - 4980 codons}

\begin{center}
    \begin{minipage}{0.325\linewidth}
        \includegraphics[width=\linewidth, page=1]{inference_simulations/obs_atgc_1_MG.pdf}
    \end{minipage}
    \llap{\raisebox{1.25cm}{\scriptsize A\hspace{4.35cm}}}\hfill
    \begin{minipage}{0.325\linewidth}
        \includegraphics[width=\linewidth, page=1]{inference_simulations/obs_atgc_2_MG.pdf}
    \end{minipage}
    \llap{\raisebox{1.25cm}{\scriptsize B\hspace{4.35cm}}}\hfill
    \begin{minipage}{0.325\linewidth}
        \includegraphics[width=\linewidth, page=1]{inference_simulations/obs_atgc_3_MG.pdf}
    \end{minipage}
    \llap{\raisebox{1.25cm}{\scriptsize C\hspace{4.35cm}}}\hfill
    \begin{minipage}{0.325\linewidth}
        \includegraphics[width=\linewidth, page=1]{inference_simulations/obs_atgc_1_MF.pdf}
    \end{minipage}
    \llap{\raisebox{1.25cm}{\scriptsize D\hspace{4.35cm}}}
    \begin{minipage}{0.325\linewidth}
        \includegraphics[width=\linewidth, page=1]{inference_simulations/obs_atgc_2_MF.pdf}
    \end{minipage}
    \llap{\raisebox{1.25cm}{\scriptsize E\hspace{4.35cm}}}
    \begin{minipage}{0.325\linewidth}
        \includegraphics[width=\linewidth, page=1]{inference_simulations/obs_atgc_3_MF.pdf}
    \end{minipage}
    \llap{\raisebox{1.25cm}{\scriptsize F\hspace{4.35cm}}}
    \begin{minipage}{0.325\linewidth}
        \includegraphics[width=\linewidth, page=1]{inference_simulations/omega_MG.pdf}
    \end{minipage}
    \llap{\raisebox{1.25cm}{\scriptsize G\hspace{4.35cm}}}
    \begin{minipage}{0.325\linewidth}
        \includegraphics[width=\linewidth, page=1]{inference_simulations/omega_MF.pdf}
    \end{minipage}
    \llap{\raisebox{1.25cm}{\scriptsize H\hspace{4.35cm}}}
\end{center}
Simulations for 61 primate taxa and 9960 codon sites, for 32 different values of mutational bias ($\lambda$) from 0.2 to 5.0 with 5 replicates per value (simulations shown in the main manuscript).
Observed $\atgc$ from the alignment and predicted $\atgc$ by the models at the first (panels A \& D), second (panels B \& E) and third (panels C \& F) codon positions as a function of mutational bias for Muse \& Gaut (panels A, B \& C) and our tensor model (panel C, D \& E).
Estimated $\hat{\omega}$ and simulated $\omega$ across replicates as a function of mutational bias ($\lambda$) for Muse \& Gaut (panel G) and our tensor model (panel H).


\subsection{Primate phylogeny - 498 codons}

\begin{center}
    \begin{minipage}{0.325\linewidth}
        \includegraphics[width=\linewidth, page=1]{inference_supp_mat/PrimatesExons1Mu1.0_lambda_MG.pdf}
    \end{minipage}
    \llap{\raisebox{1.25cm}{\scriptsize A\hspace{4.35cm}}}\hfill
    \begin{minipage}{0.325\linewidth}
        \includegraphics[width=\linewidth, page=1]{inference_supp_mat/PrimatesExons1Mu1.0_lambda_MF.pdf}
    \end{minipage}
    \llap{\raisebox{1.25cm}{\scriptsize B\hspace{4.35cm}}}\hfill
    \begin{minipage}{0.325\linewidth}
        \includegraphics[width=\linewidth, page=1]{inference_supp_mat/PrimatesExons1Mu1.0_lambda_GTR.pdf}
    \end{minipage}
    \llap{\raisebox{1.25cm}{\scriptsize C\hspace{4.35cm}}}\hfill
    \begin{minipage}{0.325\linewidth}
        \includegraphics[width=\linewidth, page=1]{inference_supp_mat/PrimatesExons1Mu1.0_omega_MG.pdf}
    \end{minipage}
    \llap{\raisebox{1.25cm}{\scriptsize E\hspace{4.35cm}}}
    \begin{minipage}{0.325\linewidth}
        \includegraphics[width=\linewidth, page=1]{inference_supp_mat/PrimatesExons1Mu1.0_omega_MF.pdf}
    \end{minipage}
    \llap{\raisebox{1.25cm}{\scriptsize F\hspace{4.35cm}}}
\end{center}
Simulations for 61 primate taxa and 498 codon sites, for 8 different values of mutational bias ($\lambda$) from 0.2 to 5.0 with 5 replicates per value.
Estimated versus true mutational bias, using a codon model in which $\omega$ is modeled as a scalar (Muse \& Gaut formalism, MG, panel A) or as a tensor (mean-field approach, panel B), or by applying a GTR nucleotide model to the 4-fold degenerate third-coding positions only (panel C).
Estimated $\hat{\omega}$ and simulated $\omega$ across replicates as a function of mutational bias ($\lambda$) for Muse \& Gaut (panel D) and our tensor model (panel E).

\includegraphics[width=\linewidth, page=1]{inference_supp_mat/PrimatesExons1Mu1.0_omega_pair_MF.pdf}
True versus estimated values of $\omega$ between pairs of amino-acids.
Vertical bars are the 95\% confidence intervals for the mean value.

\subsection{Primate phylogeny - 996 codons}

\begin{center}
    \begin{minipage}{0.325\linewidth}
        \includegraphics[width=\linewidth, page=1]{inference_supp_mat/PrimatesExons2Mu1.0_lambda_MG.pdf}
    \end{minipage}
    \llap{\raisebox{1.25cm}{\scriptsize A\hspace{4.35cm}}}\hfill
    \begin{minipage}{0.325\linewidth}
        \includegraphics[width=\linewidth, page=1]{inference_supp_mat/PrimatesExons2Mu1.0_lambda_MF.pdf}
    \end{minipage}
    \llap{\raisebox{1.25cm}{\scriptsize B\hspace{4.35cm}}}\hfill
    \begin{minipage}{0.325\linewidth}
        \includegraphics[width=\linewidth, page=1]{inference_supp_mat/PrimatesExons2Mu1.0_lambda_GTR.pdf}
    \end{minipage}
    \llap{\raisebox{1.25cm}{\scriptsize C\hspace{4.35cm}}}\hfill
    \begin{minipage}{0.325\linewidth}
        \includegraphics[width=\linewidth, page=1]{inference_supp_mat/PrimatesExons2Mu1.0_omega_MG.pdf}
    \end{minipage}
    \llap{\raisebox{1.25cm}{\scriptsize E\hspace{4.35cm}}}
    \begin{minipage}{0.325\linewidth}
        \includegraphics[width=\linewidth, page=1]{inference_supp_mat/PrimatesExons2Mu1.0_omega_MF.pdf}
    \end{minipage}
    \llap{\raisebox{1.25cm}{\scriptsize F\hspace{4.35cm}}}
\end{center}
Simulations for 61 primate taxa and 996 codon sites, for 8 different values of mutational bias ($\lambda$) from 0.2 to 5.0 with 5 replicates per value.
Estimated versus true mutational bias, using a codon model in which $\omega$ is modeled as a scalar (Muse \& Gaut formalism, MG, panel A) or as a tensor (mean-field approach, panel B), or by applying a GTR nucleotide model to the 4-fold degenerate third-coding positions only (panel C).
Estimated $\hat{\omega}$ and simulated $\omega$ across replicates as a function of mutational bias ($\lambda$) for Muse \& Gaut (panel D) and our tensor model (panel E).

\includegraphics[width=\linewidth, page=1]{inference_supp_mat/PrimatesExons2Mu1.0_omega_pair_MF.pdf}
True versus estimated values of $\omega$ between pairs of amino-acids.
Vertical bars are the 95\% confidence intervals for the mean value.

\subsection{Primate phylogeny - 2490 codons}

\begin{center}
    \begin{minipage}{0.325\linewidth}
        \includegraphics[width=\linewidth, page=1]{inference_supp_mat/PrimatesExons5Mu1.0_lambda_MG.pdf}
    \end{minipage}
    \llap{\raisebox{1.25cm}{\scriptsize A\hspace{4.35cm}}}\hfill
    \begin{minipage}{0.325\linewidth}
        \includegraphics[width=\linewidth, page=1]{inference_supp_mat/PrimatesExons5Mu1.0_lambda_MF.pdf}
    \end{minipage}
    \llap{\raisebox{1.25cm}{\scriptsize B\hspace{4.35cm}}}\hfill
    \begin{minipage}{0.325\linewidth}
        \includegraphics[width=\linewidth, page=1]{inference_supp_mat/PrimatesExons5Mu1.0_lambda_GTR.pdf}
    \end{minipage}
    \llap{\raisebox{1.25cm}{\scriptsize C\hspace{4.35cm}}}\hfill
    \begin{minipage}{0.325\linewidth}
        \includegraphics[width=\linewidth, page=1]{inference_supp_mat/PrimatesExons5Mu1.0_omega_MG.pdf}
    \end{minipage}
    \llap{\raisebox{1.25cm}{\scriptsize E\hspace{4.35cm}}}
    \begin{minipage}{0.325\linewidth}
        \includegraphics[width=\linewidth, page=1]{inference_supp_mat/PrimatesExons5Mu1.0_omega_MF.pdf}
    \end{minipage}
    \llap{\raisebox{1.25cm}{\scriptsize F\hspace{4.35cm}}}
\end{center}
Simulations for 61 primate taxa and 2490 codon sites, for 8 different values of mutational bias ($\lambda$) from 0.2 to 5.0 with 5 replicates per value.
Estimated versus true mutational bias, using a codon model in which $\omega$ is modeled as a scalar (Muse \& Gaut formalism, MG, panel A) or as a tensor (mean-field approach, panel B), or by applying a GTR nucleotide model to the 4-fold degenerate third-coding positions only (panel C).
Estimated $\hat{\omega}$ and simulated $\omega$ across replicates as a function of mutational bias ($\lambda$) for Muse \& Gaut (panel D) and our tensor model (panel E).

\includegraphics[width=\linewidth, page=1]{inference_supp_mat/PrimatesExons5Mu1.0_omega_pair_MF.pdf}
True versus estimated values of $\omega$ between pairs of amino-acids.
Vertical bars are the 95\% confidence intervals for the mean value.

\subsection{Primate phylogeny - 9960 codons}

\begin{center}
    \begin{minipage}{0.325\linewidth}
        \includegraphics[width=\linewidth, page=1]{inference_supp_mat/PrimatesExons20Mu1.0_lambda_MG.pdf}
    \end{minipage}
    \llap{\raisebox{1.25cm}{\scriptsize A\hspace{4.35cm}}}\hfill
    \begin{minipage}{0.325\linewidth}
        \includegraphics[width=\linewidth, page=1]{inference_supp_mat/PrimatesExons20Mu1.0_lambda_MF.pdf}
    \end{minipage}
    \llap{\raisebox{1.25cm}{\scriptsize B\hspace{4.35cm}}}\hfill
    \begin{minipage}{0.325\linewidth}
        \includegraphics[width=\linewidth, page=1]{inference_supp_mat/PrimatesExons20Mu1.0_lambda_GTR.pdf}
    \end{minipage}
    \llap{\raisebox{1.25cm}{\scriptsize C\hspace{4.35cm}}}\hfill
    \begin{minipage}{0.325\linewidth}
        \includegraphics[width=\linewidth, page=1]{inference_supp_mat/PrimatesExons20Mu1.0_omega_MG.pdf}
    \end{minipage}
    \llap{\raisebox{1.25cm}{\scriptsize E\hspace{4.35cm}}}
    \begin{minipage}{0.325\linewidth}
        \includegraphics[width=\linewidth, page=1]{inference_supp_mat/PrimatesExons20Mu1.0_omega_MF.pdf}
    \end{minipage}
    \llap{\raisebox{1.25cm}{\scriptsize F\hspace{4.35cm}}}
\end{center}
Simulations for 61 primate taxa and 9960 codon sites, for 8 different values of mutational bias ($\lambda$) from 0.2 to 5.0 with 5 replicates per value.
Estimated versus true mutational bias, using a codon model in which $\omega$ is modeled as a scalar (Muse \& Gaut formalism, MG, panel A) or as a tensor (mean-field approach, panel B), or by applying a GTR nucleotide model to the 4-fold degenerate third-coding positions only (panel C).
Estimated $\hat{\omega}$ and simulated $\omega$ across replicates as a function of mutational bias ($\lambda$) for Muse \& Gaut (panel D) and our tensor model (panel E).

\includegraphics[width=\linewidth, page=1]{inference_supp_mat/PrimatesExons20Mu1.0_omega_pair_MF.pdf}
True versus estimated values of $\omega$ between pairs of amino-acids.
Vertical bars are the 95\% confidence intervals for the mean value.

\subsection{Primate phylogeny - 4980 codons - branch length \%2}

\begin{center}
    \begin{minipage}{0.325\linewidth}
        \includegraphics[width=\linewidth, page=1]{inference_supp_mat/PrimatesExons10Mu0.5_lambda_MG.pdf}
    \end{minipage}
    \llap{\raisebox{1.25cm}{\scriptsize A\hspace{4.35cm}}}\hfill
    \begin{minipage}{0.325\linewidth}
        \includegraphics[width=\linewidth, page=1]{inference_supp_mat/PrimatesExons10Mu0.5_lambda_MF.pdf}
    \end{minipage}
    \llap{\raisebox{1.25cm}{\scriptsize B\hspace{4.35cm}}}\hfill
    \begin{minipage}{0.325\linewidth}
        \includegraphics[width=\linewidth, page=1]{inference_supp_mat/PrimatesExons10Mu0.5_lambda_GTR.pdf}
    \end{minipage}
    \llap{\raisebox{1.25cm}{\scriptsize C\hspace{4.35cm}}}\hfill
    \begin{minipage}{0.325\linewidth}
        \includegraphics[width=\linewidth, page=1]{inference_supp_mat/PrimatesExons10Mu0.5_omega_MG.pdf}
    \end{minipage}
    \llap{\raisebox{1.25cm}{\scriptsize E\hspace{4.35cm}}}
    \begin{minipage}{0.325\linewidth}
        \includegraphics[width=\linewidth, page=1]{inference_supp_mat/PrimatesExons10Mu0.5_omega_MF.pdf}
    \end{minipage}
    \llap{\raisebox{1.25cm}{\scriptsize F\hspace{4.35cm}}}
\end{center}
Simulations for 61 primate taxa with decreased branch lenght by a factor 2 and 4980 codon sites, for 8 different values of mutational bias ($\lambda$) from 0.2 to 5.0 with 5 replicates per value.
Estimated versus true mutational bias, using a codon model in which $\omega$ is modeled as a scalar (Muse \& Gaut formalism, MG, panel A) or as a tensor (mean-field approach, panel B), or by applying a GTR nucleotide model to the 4-fold degenerate third-coding positions only (panel C).
Estimated $\hat{\omega}$ and simulated $\omega$ across replicates as a function of mutational bias ($\lambda$) for Muse \& Gaut (panel D) and our tensor model (panel E).

\subsection{Primate phylogeny - 4980 codons - branch length x2}

\begin{center}
    \begin{minipage}{0.325\linewidth}
        \includegraphics[width=\linewidth, page=1]{inference_supp_mat/PrimatesExons10Mu2.0_lambda_MG.pdf}
    \end{minipage}
    \llap{\raisebox{1.25cm}{\scriptsize A\hspace{4.35cm}}}\hfill
    \begin{minipage}{0.325\linewidth}
        \includegraphics[width=\linewidth, page=1]{inference_supp_mat/PrimatesExons10Mu2.0_lambda_MF.pdf}
    \end{minipage}
    \llap{\raisebox{1.25cm}{\scriptsize B\hspace{4.35cm}}}\hfill
    \begin{minipage}{0.325\linewidth}
        \includegraphics[width=\linewidth, page=1]{inference_supp_mat/PrimatesExons10Mu2.0_lambda_GTR.pdf}
    \end{minipage}
    \llap{\raisebox{1.25cm}{\scriptsize C\hspace{4.35cm}}}\hfill
    \begin{minipage}{0.325\linewidth}
        \includegraphics[width=\linewidth, page=1]{inference_supp_mat/PrimatesExons10Mu2.0_omega_MG.pdf}
    \end{minipage}
    \llap{\raisebox{1.25cm}{\scriptsize E\hspace{4.35cm}}}
    \begin{minipage}{0.325\linewidth}
        \includegraphics[width=\linewidth, page=1]{inference_supp_mat/PrimatesExons10Mu2.0_omega_MF.pdf}
    \end{minipage}
    \llap{\raisebox{1.25cm}{\scriptsize F\hspace{4.35cm}}}
\end{center}
Simulations for 61 primate taxa with increased branch lenght by a factor 2 and 4980 codon sites, for 8 different values of mutational bias ($\lambda$) from 0.2 to 5.0 with 5 replicates per value.
Estimated versus true mutational bias, using a codon model in which $\omega$ is modeled as a scalar (Muse \& Gaut formalism, MG, panel A) or as a tensor (mean-field approach, panel B), or by applying a GTR nucleotide model to the 4-fold degenerate third-coding positions only (panel C).
Estimated $\hat{\omega}$ and simulated $\omega$ across replicates as a function of mutational bias ($\lambda$) for Muse \& Gaut (panel D) and our tensor model (panel E).

\subsection{Primate phylogeny - 4980 codons - branch length x4}

\begin{center}
    \begin{minipage}{0.325\linewidth}
        \includegraphics[width=\linewidth, page=1]{inference_supp_mat/PrimatesExons10Mu4.0_lambda_MG.pdf}
    \end{minipage}
    \llap{\raisebox{1.25cm}{\scriptsize A\hspace{4.35cm}}}\hfill
    \begin{minipage}{0.325\linewidth}
        \includegraphics[width=\linewidth, page=1]{inference_supp_mat/PrimatesExons10Mu4.0_lambda_MF.pdf}
    \end{minipage}
    \llap{\raisebox{1.25cm}{\scriptsize B\hspace{4.35cm}}}\hfill
    \begin{minipage}{0.325\linewidth}
        \includegraphics[width=\linewidth, page=1]{inference_supp_mat/PrimatesExons10Mu4.0_lambda_GTR.pdf}
    \end{minipage}
    \llap{\raisebox{1.25cm}{\scriptsize C\hspace{4.35cm}}}\hfill
    \begin{minipage}{0.325\linewidth}
        \includegraphics[width=\linewidth, page=1]{inference_supp_mat/PrimatesExons10Mu4.0_omega_MG.pdf}
    \end{minipage}
    \llap{\raisebox{1.25cm}{\scriptsize E\hspace{4.35cm}}}
    \begin{minipage}{0.325\linewidth}
        \includegraphics[width=\linewidth, page=1]{inference_supp_mat/PrimatesExons10Mu4.0_omega_MF.pdf}
    \end{minipage}
    \llap{\raisebox{1.25cm}{\scriptsize F\hspace{4.35cm}}}
\end{center}
Simulations for 61 primate taxa with increased branch lenght by a factor 4 and 4980 codon sites, for 8 different values of mutational bias ($\lambda$) from 0.2 to 5.0 with 5 replicates per value.
Estimated versus true mutational bias, using a codon model in which $\omega$ is modeled as a scalar (Muse \& Gaut formalism, MG, panel A) or as a tensor (mean-field approach, panel B), or by applying a GTR nucleotide model to the 4-fold degenerate third-coding positions only (panel C).
Estimated $\hat{\omega}$ and simulated $\omega$ across replicates as a function of mutational bias ($\lambda$) for Muse \& Gaut (panel D) and our tensor model (panel E).

\subsection{Primate phylogeny - 4980 codons - branch length x8}

\begin{center}
    \begin{minipage}{0.325\linewidth}
        \includegraphics[width=\linewidth, page=1]{inference_supp_mat/PrimatesExons10Mu8.0_lambda_MG.pdf}
    \end{minipage}
    \llap{\raisebox{1.25cm}{\scriptsize A\hspace{4.35cm}}}\hfill
    \begin{minipage}{0.325\linewidth}
        \includegraphics[width=\linewidth, page=1]{inference_supp_mat/PrimatesExons10Mu8.0_lambda_MF.pdf}
    \end{minipage}
    \llap{\raisebox{1.25cm}{\scriptsize B\hspace{4.35cm}}}\hfill
    \begin{minipage}{0.325\linewidth}
        \includegraphics[width=\linewidth, page=1]{inference_supp_mat/PrimatesExons10Mu8.0_lambda_GTR.pdf}
    \end{minipage}
    \llap{\raisebox{1.25cm}{\scriptsize C\hspace{4.35cm}}}\hfill
    \begin{minipage}{0.325\linewidth}
        \includegraphics[width=\linewidth, page=1]{inference_supp_mat/PrimatesExons10Mu8.0_omega_MG.pdf}
    \end{minipage}
    \llap{\raisebox{1.25cm}{\scriptsize E\hspace{4.35cm}}}
    \begin{minipage}{0.325\linewidth}
        \includegraphics[width=\linewidth, page=1]{inference_supp_mat/PrimatesExons10Mu8.0_omega_MF.pdf}
    \end{minipage}
    \llap{\raisebox{1.25cm}{\scriptsize F\hspace{4.35cm}}}
\end{center}
Simulations for 61 primate taxa with increased branch lenght by a factor 8 and 4980 codon sites, for 8 different values of mutational bias ($\lambda$) from 0.2 to 5.0 with 5 replicates per value.
Estimated versus true mutational bias, using a codon model in which $\omega$ is modeled as a scalar (Muse \& Gaut formalism, MG, panel A) or as a tensor (mean-field approach, panel B), or by applying a GTR nucleotide model to the 4-fold degenerate third-coding positions only (panel C).
Estimated $\hat{\omega}$ and simulated $\omega$ across replicates as a function of mutational bias ($\lambda$) for Muse \& Gaut (panel D) and our tensor model (panel E).

\subsection{Mammalian phylogeny - 4980 codons}

\begin{center}
    \begin{minipage}{0.325\linewidth}
        \includegraphics[width=\linewidth, page=1]{inference_supp_mat/MammalsExons10Mu1.0_lambda_MG.pdf}
    \end{minipage}
    \llap{\raisebox{1.25cm}{\scriptsize A\hspace{4.35cm}}}\hfill
    \begin{minipage}{0.325\linewidth}
        \includegraphics[width=\linewidth, page=1]{inference_supp_mat/MammalsExons10Mu1.0_lambda_MF.pdf}
    \end{minipage}
    \llap{\raisebox{1.25cm}{\scriptsize B\hspace{4.35cm}}}\hfill
    \begin{minipage}{0.325\linewidth}
        \includegraphics[width=\linewidth, page=1]{inference_supp_mat/MammalsExons10Mu1.0_lambda_GTR.pdf}
    \end{minipage}
    \llap{\raisebox{1.25cm}{\scriptsize C\hspace{4.35cm}}}\hfill
    \begin{minipage}{0.325\linewidth}
        \includegraphics[width=\linewidth, page=1]{inference_supp_mat/MammalsExons10Mu1.0_omega_MG.pdf}
    \end{minipage}
    \llap{\raisebox{1.25cm}{\scriptsize E\hspace{4.35cm}}}
    \begin{minipage}{0.325\linewidth}
        \includegraphics[width=\linewidth, page=1]{inference_supp_mat/MammalsExons10Mu1.0_omega_MF.pdf}
    \end{minipage}
    \llap{\raisebox{1.25cm}{\scriptsize F\hspace{4.35cm}}}
\end{center}
Simulations for 90 mammalian taxa and 4980 codon sites, for 8 different values of mutational bias ($\lambda$) from 0.2 to 5.0 with 5 replicates per value.
Estimated versus true mutational bias, using a codon model in which $\omega$ is modeled as a scalar (Muse \& Gaut formalism, MG, panel A) or as a tensor (mean-field approach, panel B), or by applying a GTR nucleotide model to the 4-fold degenerate third-coding positions only (panel C).

\end{document}