\thispagestyle{empty}
\chapter{Historical perspective on molecular evolution}
{\hypersetup{linkcolor=GREYDARK}\minitoc}
\label{chap:intro-historical}

From the discovery of evolution to today's knowledge, the understanding of the mechanisms by which the diversity and the complexity of living forms emerge has seen dramatic changes and has gone through several scientific revolutions.
Molecular evolutionary sciences represent one such revolution, a relatively recent scientific development emerging at the crossroads of two scientific fields.
On the one hand, evolutionary biology, which has seen tremendous theoretical development in the nineteenth and twentieth century.
On the other hand, molecular biology, which recruited the advances in biochemistry over the 20th century and has seen many technical revolutions over this time.
Being both empirical and theoretical, molecular evolution borrows strength simultaneously from the ever-increasing amount of empirical data available in molecular biology and from the predictive power of theoretical evolutionary biology.
From the differences in the observed molecular sequences between individuals of the same population, or between species, biologists can uncover the processes generating this diversity, and unravel the forces governing the underlying evolutionary mechanisms.
Can we quantify the relative strength of these forces, shaping both extant populations but also ancient and sometimes extinct lineages?
In a nutshell, molecular evolution leverages the patterns of genetic variation carried by individuals in order to uncover evolutionary mechanisms shaping the evolution of organisms and their ancestral lineages, while at the same time shedding new light on cellular and molecular processes allowing organisms to live and reproduce.

This section will recall the theoretical basis, the assumptions and the limitations on which molecular evolution is based.
It is a modest attempt, neither exhaustive nor accurate, probably imprinted with the ideology of our current society on how we perceive and interpret past discoveries.
Moreover, this introduction will highlight a few names, while in reality much of the development of molecular evolution also benefited from the contribution of many unmentioned and sometimes forgotten scientists.


\section{Population-genetics}
Molecular evolution is theoretically built upon the framework of population genetics, which in turn historically emerged as a unifying theory between Mendelian inheritance and quantitative genetics, in the early twentieth century.
Originally, Johann Gregor Mendel established the statistical laws governing heredity of discrete characters through hybridization experiments on the garden pea plant \textit{Pisum sativum} between 1857 and 1864.
This model of inheritance was rediscovered and confirmed in the early twentieth century independently by botanists Hugo de Vries, Carl Correns and Erich von Tschermak~\citep{dunn2003gregor}.

At first, models of Mendelian inheritance were deemed incompatible with the models of biometricians.
The crux of the argument revolved around the evolution of continuous characters\footnote{Incompatibility between continuous and discrete evolution can actually be traced back to debates between Jean-Baptiste de Lamarck (1744-1829) defending gradual changes and Georges Cuvier (1869-1932) supporting punctual catastrophic changes, in the late eighteenth century.}.
Broadly speaking, supporters of Mendelian genetics held that evolution was driven by mutations transmitted by the discrete segregation of {alleles}, which biometricians rejected on the basis that this would necessarily imply discontinuous evolutionary leaps~\citep{bowler2003evolution}.
Conversely, biometricians claimed that variation was continuous, which Mendelian geneticists rejected on the basis that the variation measured by biometricians was too small to be impacted by selection~\citep{provine2001origins}.

In a series of articles over the 1920s, the statistician Ronald A.\ Fisher reconciled both theories.
First, he proved mathematically that multiple discrete loci could result in a continuous variation~\citep{fisher1919xv}.
Secondly, \citet{fisher1930genetical} and \citet{haldane1932causes} proved that natural selection could change \gls{allele} frequencies in a population.
Fisher and Haldane hence articulated selection on continuous traits with discrete underlying genetic inheritance, a work that was completed by \citet{wright1932roles} for combinations of interacting genes.
Wright also proposed the concept of fitness landscape, viewing the evolution of a population as a hill-climbing process.
In this context, Wright also explored some of the consequences of random drift, proposing that drift could sometimes allow for a population to cross a valley between multiple fitness peaks (refs).
Altogether, Fisher, Haldane and Wright laid the foundations of population genetics, a discipline which basically integrated Mendelian genetics, Darwinism and biometry, easing the debate between continuous and gradual evolution\footnote{This debate was revived by paleontologists \citet*{Gould1972}.
As of today it is admitted that both macroevolutive patterns of punctual and gradual changes can be found.}.

The emergence of this new scientific field was the first step towards the development of a unified theory of evolution named the ‘modern synthesis’~\citep{huxley1942evolution}, essentially defined on the basis that natural selection acts on the heritable variation supplied by mutations~\citep{mayr1959where,stebbins1966processes,dobzhansky1974chance}.


\section{Central dogma of molecular biology}

During the theoretical development of population genetics, the support of heredity was largely unknown, and the terminology of 'gene', '\glspl{allele}' and 'locus' was essentially theoretical and not grounded on directly observable correlates.
The first evidence that deoxyribonucleic acid (\acrshort{DNA}) is the molecular support of genetic information is in the work of \citet{Avery1944}, who showed that bacteria treated with a deoxyribonuclease enzyme failed to transform, while otherwise transforming when treated by a protease.
The chemical composition of \acrshort{DNA} was further elucidated by \citet{Chargaff1950}, who found that the proportions of adenine (A) and thymine (T) in \acrshort{DNA} were roughly the same as the amounts of cytosine (C) and guanine (G), suggesting a relation of complementarity between base pairs (A:T and G:C).
On the other hand, the proportion of G+C was found to vary from one species to another, which provided evidence that \acrshort{DNA} could encode the genetic information, via a four-letter molecular alphabet.

Ultimately, the double-helix structure of \acrshort{DNA} was deciphered by \citet{franklin1953molecular}, \citet{watson1953molecular} and \citet{wilkins1953molecular}.
Once the molecular structure of DNA and its role as a support of heredity was elucidated, the work of \citet{Crick1958} on the question of the transfer of information from DNA to proteins resulted in the determination of the genetic code, the translation table from triplets of nucleotides (codons) to amino acids.
Ultimately, the establishment of the central dogma of molecular biology detailed the process of protein synthesis~\citep{Crick1970}.
Briefly, the central dogma of molecular biology states that the \textit{determination of sequence from nucleic acid to nucleic acid, or from nucleic acid to protein may be possible, but transfer from protein to protein, or from protein to nucleic acid is impossible}.

As the support of heredity, DNA gained a central role in evolutionary biology.
Moreover, the development of new technologies such as the polymerase chain reaction (PCR), Sanger sequencing and their subsequent refinements revolutionized the availability of empirical data on which to test the theoretical predictions of population genetics.


\section{Neutral theory}

Although a unifying theory, population-genetics remained rather theoretical for some time, because it deals with the concept of gene frequencies, yet there was no direct way to unambiguously identify the genes with the observable traits at the phenotypic level.
For that reason, the connection between theoretical population genetics and empirical and experimental work was only indirect, although quite precisely formalized, through quantitative genetics.
Quantitative genetics, or the genetics of complex traits, works by proposing a ‘microscopic’ model of the genetic architecture of a given observable phenotypic trait.
This entails the specification of the number of loci, the effect sizes contributed by each of them, the possible dominance or epistatic interactions between alleles at the same locus or between loci, etc.
Population genetics is then used to derive theoretical expectations about the response of the trait to artificial or natural selection, predictions which are then tested against empirical data~\citep{Lande1976,Lande1980,Lande1983}.
In this framework, however, the detailed genetic basis of the evolutionary process is never accessed directly, but is only indirectly tested.

The situation changed radically during the second half of the 20th century.
With the advent of molecular genetics, it became possible to have a direct access to the variability of nucleic and protein sequences within a species, as well as to the differences between closely related species, making it possible to estimate the rate at which allelic genes are substituted.
The new observations that were made thanks to these new technological developments turned out to create some surprise.

First, by comparing protein sequences from related species, it was observed that the number of point substitutions between pairs of species was approximately proportional to the time since their last common ancestor~\citep{Zuckerkandl1965,Salser1976}.
These observations led to posit the molecular clock hypothesis, which assumes that the rate at which point substitutions accumulate is approximately constant through time.
This apparently constant rate of molecular evolution is in sharp contrast with the much more variable rate of morphological evolution observed in the same species, and more generally across the entire fossil record~\citep{Simpson1944,Simpson1953}.
Second, electrophoretic methods uncovered surprisingly high levels of genetic variability within natural populations, such that most proteins in diverse organisms were found to be naturally polymorphic~\citep{Harris1966, Hubby1966, Lewontin1966}.
In many cases, this molecular polymorphism had no visible phenotypic effects and showed no obvious correlation with any other covariate.
Finally, by comparing DNA sequences in related species, it was observed that the overall (genome-wide) rate of DNA substitutions is very high, of least one nucleotide base per genome every two years in a mammalian lineage.

These observations are not easily explained in purely adaptive terms.
Instead, they led \citet{Kimura1968}, and independently, \citet{King1969}, to propose the \gls{neutral} theory of molecular evolution~\citep{kimura1986dna, kimura1991neutral}.
The main tenet of the neutral theory is that most intra- and inter-specific molecular variation is in fact adaptively neutral, thus explaining the high protein variability observed in polymorphism datasets, where the diversity is supplied by a high mutational input.
Subsequently to origination by mutation, this selectively neutral diversity is reduced by the random extinction of alleles, via the cumulative effect of the random sampling of alleles at each generation.
Although the likely fate of a neutral allele just created by mutation is its ultimate extinction, it is also possible that random drift leads to the fixation of this allele in the population.
In this context, the frequency of the neutral allele fluctuates through generations, randomly increasing or decreasing over time, because only a relatively small number of gametes are randomly sampled out of the vast number of male and female gametes produced in each generation.
As a consequence, the effect of genetic drift at the level of a population results into divergence between lineages, where the majority of the nucleotide substitutions in the course of evolution must have been the result of the random fixation of neutral mutants rather than the result of positive Darwinian selection.
Of note, the neutral theory does not say that most mutations are neutral or that adaptation does not take place.
A substantial fraction of all mutations are in fact strongly deleterious.
However, those mutations are quickly purified away and are generally not visible, either in the polymorphism within species or in the divergence between species.
The argument of the neutral theory is just that most mutations that are not deleterious are essentially neutral.
Adaptive mutations are just rare, relative to neutral mutations, and as a consequence, adaptive arguments do not need to be invoked in order to explain most of the observed intra- and inter-specific variation.

In a second step, \citet{Ohta1971} refined the neutral theory, by proposing that mutations can have an effect on phenotype, and therefore on fitness.
However, if their effect on fitness is sufficiently small, they should still behave neutrally and have their fate dictated solely by drift.
\citet{ohta1973slightly} later proposed a mathematical formalization of this argument, incorporating weakly selected mutations to propose the \gls{nearly-neutral} theory.
This theory emphasizes that selective effects lower than the inverse of effective population size are negligible and are expected to behave neutrally.
In this regard, effective population size ($\Ne$) is a quantitative measure of drift, such that genetic drift decreases with increased effective population size.

The neutral theory sparked a long-standing controversy between neutralist and selectionists.
Selectionists maintain that a mutant \gls{allele} must have some selective advantage to spread through a species, although admitting that a \gls{neutral} \gls{allele} may occasionally be carried along by hitchhiking on a closely linked gene that is positively selected.
Neutralists, on the other hand, argued that some mutants might spread through a population without having any selective advantage, just by random sampling, such that if a mutant is selectively equivalent to preexisting resident \glspl{allele}, its fate is thus left to chance.
Of note, even if the probability of fixation of any given neutral mutation is low ($p = 1/2\Ne$), the high rate of mutation at the gene or genome-wide level and the highly degenerate mapping between genotype and phenotype both leaves considerable latitude at the molecular level for random genetic changes that have no effect upon the fitness of the organism~\citep{King1969}.
As a result, the total flux of neutral substitutions can in fact be the dominant contribution to intra-specific polymorphism and inter-specific differences.
This overwhelming combinatorial effect was probably the point that was hard to grasp by many evolutionary biologists at the time, trained in the idea that most mutations should have an effect on the phenotype.
Another factor that contributed to the difficulty in accepting the neutral theory is the fact that effective population sizes turn out to be much smaller than true (census) population sizes.
This point is important, because, according to the nearly-neutral theory of \citet{Ohta1992}, the inverse of effective population size directly determines the proportion of all mutations that are effectively neutral.
Once it is recognized that effective population sizes are small, it becomes easier to accept that most mutations with weak effects are effectively neutral.

As of today, it is widely accepted that both \gls{drift} and natural selection participate in the evolution of genomes.
The controversy is no longer strictly dichotomous but rather concerns the quantitative contributions of adaptive and of non-adaptive evolutionary processes, and their articulation with regards to mutation, selection, drift, migration, \gls{GeneConv}, and other evolutionary processes.


\section{The legacy of the nearly-neutral theory}
\label{sec:the-legacy-of-the-nearly-neutral-theory}

The neutral theory, and its nearly-neutral extension, have broad implications in evolutionary biology.
Much of its insight has been integrated in modern population genetics, molecular evolutionary sciences, but also phylogenetics and molecular dating.
Importantly, because of the marginal role played in this theory by the most unpredictable factor involved in molecular evolution, namely adaptation, the nearly-neutral theory is in a good position to make clear quantitative predictions about the rate and patterns of molecular evolution, or about the structure of genetic diversity within species.
As such it gives a well-defined framework to formalize various assumptions about the underlying processes and test them against empirical sequence data, which are becoming increasingly available.
Questions within this framework range from the causes of mutational rate variation, to the structure of fitness landscapes, or the impact of changes in effective population size between species.
In the following, I summarize several of the most important insights that have been contributed by the neutral and nearly-neutral theory, and how they still play on current research in molecular evolution.

\subsection{Mostly-purifying selection}
\label{subsec:mostly-purifying-selection}

First, along with the adoption of the nearly-neutral theory by evolutionary biologists, the common perception about the nature of selection shifted from selection being a driver of changes mediated by adaptive mutations to a mainly purifying force discarding and filtering out strongly deleterious mutations~\citep{Lynch2007}.
From this perspective, protein sequences are relatively close to their adaptive optimum, such that many mutations occurring in their sequence are likely to disrupt their functions.
This effect can be observed in underlying DNA sequences, where non-synonymous substitutions occur less frequently than synonymous substitutions~\citep{King1969}, and similarly, radical amino acid replacements are more frequent than conservative changes~\citep{kimura1983neutral}.
These effects are also observed within populations, non-synonymous single-nucleotide polymorphisms segregate at lower frequencies compared to synonymous polymorphisms, a phenomenon explained by purification of deleterious alleles which cannot reach high frequencies~\citep{Akashi1999, Cargill1999, Hughes2005}.
Finally, what determines the rate of non-synonymous evolution of protein-coding genes is primarily the amount of selective constraint acting on them, such that slowly evolving genes are just more constrained than fast-evolving genes~\citet{kimura1983neutral}.

\subsection{The mutation-selection balance}
\label{subsec:the-mutation-selection-balance}

Proteins are relatively close to, but not quite at their optimum.
This relates to another important conceptual point contributed by the nearly-neutral theory.
From a neutralist perspective, evolution should not be seen as an optimization process, but instead, as a process driving natural protein sequences at their mutation-selection equilibrium.
This concept of mutation-selection balance explains important features of natural protein sequences, which cannot be explained only in terms of optimization.
Thus, as noted early on by \citet{King1969}, amino acids that have more codons are more frequently represented in natural protein coding sequences.
Similarly, later work by \citet{Singer2000} has shown that species with a mutational bias towards AT (respectively GC) tended to have proteomes with a higher frequency of amino acids encoded by AT-rich (respectively GC-rich) codons.
Another implication is that proteins are not optimal, either for their enzymatic properties~\citep{Cornish-Bowden1976,Albery1976,Hartl1985} or for their conformational stability~\citep{Taverna2002}.
This non-optimality is observed even if proteins are under directional selection for the optimal sequence.
All these observations are clear illustrations of the fact that natural sequences are not at their optimum, but instead, are the result of a trade-off between mutation biases and mostly purifying selection.
This trade-off between mutation and selection is regulated by the amount of random drift, and thus by effective population size.
The concept of mutation-selection balance is not yet fully incorporated in evolutionary thinking.
Many evolutionary scientists, and many biologists more generally, still tend to think in terms of optimization.
Correctly formalizing this interplay between mutation, selection and drift in the context of phylogenetic codon models is in fact at the core of most of the work presented in the thesis.

\subsection{The importance of drift}
\label{subsec:importance-of-drift}

Tempering the effect of selection, drift mediated by effective population size has been repeatedly invoked to explain the relaxation of the selective strength.
First, it has been observed that within populations relative diversity of selected site is more reduced for species with smaller effective population size.
Indeed, in an intra-specific context, the non-synonymous diversity, relative to the synonymous diversity (i.e. $\pnps$), is reduced in species characterized by smaller effective population sizes~\citep{Piganeau2009, Elyashiv2010, Galtier2016, Chen2017, James2017}.
Similarly, in a phylogenetic context, the strength of selection, such as measured by the relative rate of non-synonymous over synonymous substitution, is lower along lineages with small effective population size~\citep{Ohta1993, Tomoko1995a, Moran1996, Woolfit2003, Woolfit2005, Popadin2007}.
It is important to note that, in most cases, the effective population size is not directly measured, but a surrogate measure is used instead, for example synonymous diversity ($\ps$), body size or longevity.
Leveraging the nearly-neutral theory in order to quantitatively measure effective population size in a phylogenetic context is one of the main objectives of this thesis, such as presented in chapter~\ref{chap:MutSelDrift}.
Of note, the quantitative response of the molecular evolutionary process to changes in effective population size appears to strongly depend on the underlying fitness landscapes~\citep{Welch2008}, to the point of being entirely absent~\citep{Cherry1998, Goldstein2013}.
This relationship between the rate of evolution and effective population size is also a main question addressed in this thesis, such as studied in chapter~\ref{chap:GenoPhenoFit}.

\subsection{Where is adaptation?}
\label{subsec:where-is-adaptation?}

The neutralist view of selection as mostly purifying raises an important question: where, and to what extent, does adaptation leave traces in molecular sequences?
The fact that the neutral theory has been relatively silent on this question has largely contributed to its rejection by many biologists, and in many respects the question is still open.
At first, methods for detecting adaptation have been developed, integrating either the neutral or the nearly-neutral regime as a null model.
Departures from one of these null model are then typically interpreted as traces of adaptations.
This idea to detect traces of adaptation has been explored in a phylogenetic context, whenever the null model is neutral~\citep{Goldman1994, Muse1994, Yang2002, Zhang2004} or nearly-neutral~\citep{Rodrigue2016, Bloom2017}.
Similarly, in a population-genetics context, adaptation is detected as a deviation from the null model, considered originally neutral~\citep{McDonald1991, Charlesworth1994, Smith2002}, and subsequently improved to account for slightly deleterious mutations in a nearly-neutral regime~\citep{eyre-walker_estimating_2009, Galtier2016}.

These methods have clearly revealed important traces of adaptation~\citep{Bustamante2005, Halligan2010, Enard2014}, in particular, in genes implicated in host-pathogen interactions~\citep{Enard2016, Grandaubert2019}, or in other specific genes involved in intra-genomic Red-Queen dynamics such as PRDM9~\citep{Thomas2009,Oliver2009,Ponting2011}.
However, this might represent only the most extreme adaptive events.
Much of adaptation might still have been missed at the molecular level.
\citet{kimura1983neutral} proposed a more radical insight about the link between phenotypic adaptation and neutral molecular evolution.
By showing an example of a phenotypic trait under stabilizing selection and controlled by a large number of loci with small effects, phenotype efficiently optimized by selection, but the molecular evolutionary process at each locus essentially indistinguishable from a neutral process.
More recent work, using the empirical knowledge acquired by large-scale population-genomics project in humans, draws similar conclusions~\citep{Simons2018}.
Namely that many traits turn out to be highly polygenic~\citep{Pritchard2002}, and the frequency changes contributing to their adaptive fine-tuning can be highly stochastic~\citep{Sella2019}.
Analogous to statistical physics, microscopic behavior of a physical system is dominated by thermal noise, while the macroscopic state looks essentially deterministic and driven by a principle of free-energy minimization.

\subsection{Molecular evolution is mutation-limited}
\label{subsec:molecular-evolution-is-mutation-limited}

Originally, the neutral theory was heavily relying on the molecular clock hypothesis of \citet{Zuckerkandl1965}, which posits that rate of sequence evolution is constant through time and across evolutionary lineages.
Although appealing, it became clear that the rate of evolution was not constant~\citep{ChungWu1985, Li1987, Bulmer1991, Gaut1992}.
This rejection of the strict clock motivated important methodological developments for modeling the fluctuations of the substitution rate along a phylogeny~\citep{Sanderson1997, Thorne1998, Kishino2001, Aris-Brosou2002, Drummond2006, Lepage2007}.
The primary motivation for these relaxed clock models was to achieve more accurate molecular dating.
However, these developments also fostered comparative analyses, trying to explain the causes of the variation of substitution rate between lineages.
Methodologically, this motivated the developments of methods able to conduct correlation analyses between molecular evolutionary rates and observable quantitative traits, while correcting for phylogenetic inertia~\citep{Lanfear2010a, Lartillot2011}.
Empirically, generation-time, but also metabolic rate, or selection for longevity, are potential explanations for the variation in substitution rate~\citep{Lartillot2012}, which can be interpreted in the light of the molecular mechanisms of cell division~\citep{Gao2016}.

The exact reasons for the variation in substitution rate between lineages is still debated.
However, what is clear is that this variation is mostly reflecting variation in the mutation rate.
As such, and in spite of the historically central role played by the molecular clock in the arguments in favor of the neutral theory, the rejection of the molecular clock by empirical data does not contradict the neutral theory.
It just confirms that, in a neutral or nearly-neutral regime, the molecular evolutionary process is mutation-limited, or in other words that the substitution rate is determined primarily by the mutation rate.

\subsection{Extending the null-hypothesis of molecular evolution}
\label{subsec:extending-the-null-hypothesis-of-molecular-evolution}

Finally, some patterns have been found inconsistent within the general framework of mutation, selection and drift, thus leading to uncovering new forces such as biased gene conversion which mimics selection but are fundamentally segregation distortion during recombination~\citep{Marais2003,Galtier2007,Duret2009}.
Such forces are altering the composition of genomes and must be carefully accounted for in models of evolution~\citep{Galtier2009,Ratnakumar2010, Figuet2014}.
However, even though forces such as biased gene conversion are not within the scope of this thesis, some assumptions and design of our models had been taken such as to implement these forces subsequently.

\subsection{Conclusion}
\label{subsec:conclusion}

Altogether, evolution of sequences result from the interplay between mutation, selection and drift, where this formalism is developed in chapter~\ref{chap:intro-formalism}.
Of all these components, selection is the most pervasive, which can be approximated and observed in protein-coding \acrshort{DNA} sequences in a phylogenetic context between lineages, presented in chapter~\ref{chap:intro-codon-models}).
Consequently, models are applied to empirical data, and the methodology of Bayesian inference from an alignment of DNA sequences is presented in chapter~\ref{chap:intro-inference}.
Finally, selection of protein coding DNA sequences is related to bio-chemical and bio-physical constraints (chapter~\ref{chap:intro-inference}).
