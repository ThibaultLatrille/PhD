\begin{center}
	\huge
	\textbf{Modelling the interplay between selective and neutral mechanisms in the evolution of protein-coding DNA sequences}
\end{center}

\section*{Abstract}

Molecular evolution aims to characterize the mechanisms at work in the evolution of genetic sequences.
This evolution is governed by a stochastic process whose main components are mutation, selection and genetic drift.
In the long term, this stochastic process results in a history of substitution events along species trees, inducing complex patterns of molecular divergence between species.
By analysing them, phylogenetic codon models aim at capturing the intrinsic parameters of evolution.
In this context, this thesis has been focused on phylogenetic codon models, and on how they can be used to understand the interplay between mutation, selection and drift in shaping protein-coding DNA sequences.
Because the composition of protein-coding DNA sequences does not reflect the underlying mutational process, but its filtering by selection at the level of amino acids, a careful modelling approach is necessary to tease apart mutation and selection.
Current codon models are inherently misspecified in this respect and, as a result, do not return accurate estimates of mutation biases.
Therefore, I first developed a phylogenetic codon model in which the ratio of the non-synonymous over the synonymous substitution rates is modelled as a tensor, rather than a scalar.
This model gives a more accurate representation of how mutation and selection oppose each other at equilibrium and yields accurate estimates of the mutation bias.
Second, the balance between the opposing forces of mutation and selection is arbitrated by genetic drift, which in turn is modulated by effective population size.
As a consequence, variation in effective population size along of a phylogeny can theoretically be inferred from the trails of substitutions along the lineages.
I thus developed a second model of inference, jointly reconstructing site-specific fitness landscapes and the variation in effective population size and in the mutation rate along the phylogeny.
This Bayesian framework was tested against simulated data and then applied to empirical data.
Estimated lineage-specific ancestral population sizes show the expected correlation with life-history traits or ecological variables.
However, the magnitude of the inferred variation is narrower than expected based on independent estimates.
In order to understand this narrow variation in the estimated effective population sizes, and the possible role of epistasis in this outcome, i finally developed a theoretical model describing how changes in both effective population size or expression level of protein translate into a change in the substitution rate, and how this response depends on the underlying sequence-phenotype-fitness map.
I more specifically explored a biophysical model assuming that proteins are under directional selection to maximize their conformational stability.
Results of this theoretical and simulation work imply a weak response (or susceptibility) of the substitution rate to changes in expression level or effective population size (which are interchangeable).
Theoretical approximations were also developed, expressing this susceptibility as a function of the parameters of the biophysical model.
Finally, these quantitative estimates are discussed in the light of current empirical knowledge.
Altogether, this thesis demonstrates that the assumptions made on the structure of the fitness landscape have a critical importance on the sensitivity of the substitution rate to changes in ecological or molecular variables.
Conversely, empirical observations of the patterns of substitutions in response to changes in molecular or ecological variables inform us about the underlying structure of the fitness landscape.
Being based on the mutation-selection balance and by explicitly integrating effective population size, my work also presents a conceptual framework relating phylogenetic and population genetics, while proposing conceptual and methodological paths in order to achieve their unification.
