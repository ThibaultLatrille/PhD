The first strategy is to augment information about interspecies conservation with information about genetic polymorphism.
$g(x, \scaledselcoef) \der x $ is the expected time for which the population frequency of the derived \gls{allele}, at the site, is in the range $(x, x+\der x)$ before eventual absorption:
\begin{align}
	g(x, \scaledselcoef) & \approx \dfrac{2 \left[ 1 - \e^{-\scaledselcoef (1-x)}\right]}{(1 - \e^{-\scaledselcoef})x(1-x)}
\end{align}
\citet{Sawyer1992} expanded the modeling of site evolution to multiple sites.
The model makes the following assumptions:
\begin{itemize}
	\setlength\itemsep{-0.2em}
	\item Mutations arise at Poisson times (rate $u$ per site per generation)
	\item Each mutation occurs at a new site (infinite sites, irreversible)
	\item Each mutant follows an independent Wright-Fisher process (no linkage)
\end{itemize}
In a sample of size $n$, the expected number of sites with $k$ (which ranges from $1$ to $n-1$) copies of the derived \gls{allele} is defined as a function of $g(x)$:
\begin{align}
	G(i, n, \theta, \scaledselcoef) & = 2 \Ne u \int_{0}^{1} g(x, \scaledselcoef) \binom{n}{i} x^{i} (1-x)^{n-i} \der x \nonumber \\
	& = \theta \int_{0}^{1} \dfrac{1 - \e^{-\scaledselcoef (1-x)}}{(1 - \e^{-\scaledselcoef})x(1-x)} \binom{n}{i} x^{i} (1-x)^{n-i} \der x\text{, where } \theta=4\Ne u \nonumber \\
	& = \binom{n}{i} \dfrac{\theta }{1 - \e^{-\scaledselcoef}} \int_{0}^{1} \left( 1 - \e^{-\scaledselcoef (1-x)} \right) x^{i-1} (1-x)^{n-i-1} \der x
\end{align}
In the mutation selection-framework developed, the fitness of a given genotype is a function of the encoded amino-acid through the site-wise amino-acid fitness profiles ($ \Fit\siteexp $ at site $\site$). Thus the coefficient ($\scaledselcoef=4\Ne \selcoef $) associated to a mutation is a function of the amino-acids encoded by the ancestral ($\ci$) and derived ($\cj$) \gls{codon}. Altogether the selection coefficient from $\ci$ to $\cj$ at site $\site$ is:
\begin{align}
	\scaledselcoef_{\itoj}(\Ne, \Fit\siteexp) &= 4 \Ne (f_\cj\siteexp-f_\ci\siteexp) \nonumber \\
	& = \scaledfit_\cj\siteexp-\scaledfit_\ci\siteexp
\end{align}
Similarly, the mutation rate between by the ancestral ($\ci$) and derived ($\cj$) \gls{codon} is a function of the nucleotide changes between the \glspl{codon}. If the \glspl{codon} are not neighbor, meaning a single mutation is not sufficient to jump from $\ci$) to $\cj$, the mutation rate is equal to $0$. If the \glspl{codon} are neighbors, the mutation rate is given by the nucleotide rate matrix ($ \bm{u} $). Altogether, the scaled mutation rate $\theta_{\itoj}$ from $\ci$ to $\cj$ is:
\begin{equation}
	\theta_{\itoj}(\Ne, u, \Mutmatrix) = 4 \Ne u \mutmatrix_{\itoj}
\end{equation}
If a site is \gls{polymorphic} and the ancestral ($\ci$) and derived ($\cj$) \glspl{codon} are neighbors, the probability of observing $i$ copies ($n \geq i > 0$) of the derived \gls{codon} ($\cj$), in a sample of size $n$, at site $\site$, is given by:
\begin{equation}
	P(\ci=n-i,\cj=i \ |\ \Ne, u, \Mutmatrix, \Fit\siteexp) = G\left(i, n, \theta_{\itoj}(\Ne, u, \Mutmatrix), \scaledselcoef_{\itoj}(\Ne, \Fit\siteexp) \right)
\end{equation}
Moreover the probability that a site is monomorphic is given by:
\begin{equation}
	P(\ci= n \ |\ \Ne, u, \Mutmatrix, \Fit\siteexp) = 1 - \sum\limits_{\cj \in \Ni} \sum\limits_{i=1}^{n} G\left(i, n, \theta_{\itoj}(\Ne, u, \Mutmatrix), \scaledselcoef_{\itoj}(\Ne, \Fit\siteexp)\right)
\end{equation}
And all other probabilities equal to $0.0$.

$\bullet$ This method has been implemented into BayesCode and proved to be computationally intensive, even though optimizing the computation with sufficient statistics.

$\bullet$ Moreover, it doesn't solve the issue of the site specific fitness landscape.

$\bullet$ As a result, altough theoretically interesting in this form, such model would have limited practical application.
