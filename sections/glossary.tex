\newacronym{AA}{AA}{Amino-Acid}
\newacronym{AIC}{AIC}{Akaike information criterion}
\newacronym{A}{A}{Adenine}
\newacronym{HB}{HB}{Halpern \& Bruno}
\newacronym{gBGC}{gBGC}{GC Biased Gene Conversion}
\newacronym{dBGC}{dBGC}{DSB driven Biased Gene Conversion}
\newacronym{BGC}{BGC}{Biased Gene Conversion}
\newacronym{C}{C}{Cytosine}
\newacronym{CUB}{CUB}{Codon Usage Bias}
\newacronym{CDS}{CDS}{Coding DNA Sequence}
\newacronym{DSB}{DSB}{Double strand Break}
\newacronym{DFE}{DFE}{Distribution of Fitness Effects}
\newacronym{DNA}{DNA}{NeoxyriboNucleic Acid}
\newacronym{G}{G}{Guanine}
\newacronym{GTR}{GTR}{General Time Reversible}
\newacronym{GY}{GY}{Goldman \& Yang}
\newacronym{LHT}{LHT}{Life-History Traits}
\newacronym{LRT}{LRT}{Likelihood Ratio Test}
\newacronym{MCMC}{MCMC}{Markov Chain Monte Carlo}
\newacronym{MC}{MC}{Markov Chain}
\newacronym{MG}{MG}{Muse \& Gaut}
\newacronym{MK}{MK}{McDonald \& Kreitman}
\newacronym{ML}{ML}{Maximum Likelihood}
\newacronym{Ne}{\ensuremath{\Ne}}{effective population size}
\newacronym{RNA}{RNA}{RiboNucleic Acid}
\newacronym{S}{S}{Strong nucleotide (G or C)}
\newacronym{SFS}{SFS}{Site Frequency Spectrum}
\newacronym{T}{T}{Thymine}
\newacronym{tRNA}{tRNA}{transfer RNA}
\newacronym{W}{W}{Weak nucleotide (A or T)}
\newacronym{WS}{WS}{Mutation from a ‘weak’ (W) to ‘strong’ (S) nucleotide}

\newglossaryentry{allele}{name={allele},description={A variant form of a given gene}}
\newglossaryentry{GC-biased gene conversion}{name={GC-biased gene conversion},description={The process by which the GC-content increases because of biased gene conversion}}
\newglossaryentry{biased gene conversion}{name={biased gene conversion},description={Process by which gene conversion is biased towards a given outcome. It occurs when one haplotype has a higher probability of being the donor}}
\newglossaryentry{codon usage bias}{name={codon usage bias},description={Unequal frequency of the alternative codons that specify the same amino acid}}
\newglossaryentry{codon}{name={codon},description={Sequence of three nucleotides coding for a given amino acid}}
\newglossaryentry{diploid}{name={diploid},description={Organism (or phase) displaying a ploidy of 2 ($n=2$), i.e.\ two sets of chromosomes (which are paired)}}
\newglossaryentry{effective population size}{name={effective population size},description={The number of individuals in a population who contribute to the next generation}}
\newglossaryentry{GC-content}{name={GC-content},description={The percentage of G or C nucleotidic bases in a DNA sequence}}
\newglossaryentry{gamete}{name={gamete},description={Product of meiosis}}
\newglossaryentry{gene conversion}{name={gene conversion},description={A non-reciprocal recombination process that results in one sequence being converted into the other}}
\newglossaryentry{genetic drift}{name={genetic drift},description={The random fluctuation in allele frequencies due to random sampling of individuals}}
\newglossaryentry{genetic distance}{name={genetic distance},description={Distance between DNA markers on a chromosome measured as the amount of crossing-overs between them}}
\newglossaryentry{genetic interference}{name={genetic interference},description={The fact that the formation of a recombination event can affect that of others in adjacent regions}}
\newglossaryentry{genetic linkage}{name={genetic linkage},description={Non-independent assortment of genes}}
\newglossaryentry{genotyping}{name={genotyping},description={The process by which DNA is analyzed to determine which genetic variant (allele) is present for a given marker}}
\newglossaryentry{haploid}{name={haploid},description={Organism (or phase) displaying a ploidy of 1 ($n=1$), i.e.\ a single set of chromosomes}}
\newglossaryentry{N-terminus}{name={N-terminus},description={End of an amino acid chain terminated by a free amine group}}
\newglossaryentry{non-synonymous}{name={non-synonymous},description={transition that modifies the amino acid produced}}
\newglossaryentry{phenotype}{name={phenotype},description={The composite of observable traits}}
\newglossaryentry{ploidy}{name={ploidy},description={The number of complete sets of chromosomes ($n$) in a cell}}
\newglossaryentry{polymorphic}{name={polymorphic},description={Which presents several forms. Subject to inter-individual variability}}
\newglossaryentry{recombination}{name={recombination},description={Exchange of DNA sequence information}}
\newglossaryentry{synonymous}{name={synonymous},description={substitution that does not modify the amino acid produced}}
\newglossaryentry{Akaike information criterion}{name={Akaike information criterion},description={Measure of the relative quality of a maximum likelihood estimated given the data, penalizing for too many parameters in the model}}
\newglossaryentry{Bayes factor}{name={Bayes factor},description={Ratio of the posterior probability of two competing hypotheses, usually a null and an alternative. The posterior probability is conditioned on randomly observed data and on the prior distribution of the parameters of the competing hypotheses}}
\newglossaryentry{Dirichlet process}{name={Dirichlet process},description={Family of stochastic processes whose realizations are probability distributions. In other words, a Dirichlet process is a probability distribution whose range is itself a set of probability distributions. It is often used in Bayesian inference to describe the prior knowledge about the distribution of random variables. Meaning, how likely it is that the random variables are distributed according to one or another particular distribution}}
\newglossaryentry{Likelihood ratio test}{name={Likelihood ratio test},description={Statistical test used to compare the goodness of fit of two models, one of which (the null model) is a special case of the other (the alternative model)}}
\newglossaryentry{likelihood}{name={likelihood},description={Probability of observing the data given the parameters of the statistical model. Note this is a function of solely the parameters of the model, the observed data are fixed}}
\newglossaryentry{Markov chain Monte-Carlo}{name={Markov chain Monte-Carlo},description={Class of algorithms for sampling from a probability distribution (usually posterior distribution in Bayesian inference) based on constructing a Markov chain that has the desired distribution of its equilibrium distribution}}
\newglossaryentry{Markov chain}{name={Markov chain},description={Stochastic process with property that the next state of the process depends only on the present state of the process and not on its past}}
\newglossaryentry{maximum likelihood}{name={maximum likelihood},description={Method of estimating the parameters of a model given the data, by finding the parameter value that maximize the likelihood function}}
\newglossaryentry{mixture model}{name={mixture model},description={Probabilistic model for representing the presence of subpopulations within an overall population, without requiring that an observed data set should identify the sub-population to which an individual observation belongs}}
\newglossaryentry{nearly-neutral}{name={nearly-neutral},description={Slightly deleterious or advantageous mutations are effectively neutral when their selection coefficient are lower than one divided by the effective population size}}
\newglossaryentry{neutral}{name={neutral},description={Mutations are nor deleterious neither advantageous, their probability of fixation is one divided by twice the effective population size (for diploids)}}
\newglossaryentry{prior}{name={prior},description={Probability distribution that would express one's beliefs about a parameter of the model before the data is taken into account}}
\newglossaryentry{posterior}{name={posterior},description={Probability distribution of a parameter of the model conditioned on randomly observed data and the prior distribution}}
\newglossaryentry{substitution}{name={substitution},description={Point mutation that appeared in only one individual in the population, and subsequently reached fixation in the population}}

\printglossary[type=\acronymtype]
\mtcaddchapter
\printglossary
\mtcaddchapter