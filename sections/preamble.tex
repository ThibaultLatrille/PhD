\chapter*{Preamble}
\addcontentsline{toc}{part}{Preamble}
The diversity of living organisms today is the result of a complex and intricate process, which operates at multiple level.
At the molecular level, the fate of protein depends on its ability to fold but also the enzyme it encounters from its creation up to its degradation. 
Composed of billions of proteins, a cell own fate depends on its own ability to metabolize substrates and copy its \acrshort{DNA}, but also depends on the fate of surrounding cells and the individual on which it belongs.
Moreover, the fate of this individual depends on its own behavior, but also depends on its environment and the population on which it belongs.
Altogether, scientists have dissected this intricate process into its core components, trough molecular biology, enzymology, metabolism, physiology, population-genetics, ecology, and so on. 
Molecular evolution seek to encompass different levels, relating molecular changes to higher level evolutionary process.
In this vain, this work is a modest attempt to reconcile several layers of evolution, mechanistically deriving how observable parameters between population and within population depends in microscopic molecular and cellular parameters.
Altogether, through the framework of population genetic, I seek to draw connections between independent dataset, from molecular parameters of protein biophysics, to diversity and divergence of \acrshort{DNA} sequences within and between species, while relating to species quantitative life-history traits.\\

This thesis is submitted in partial fulfillment of the requirements for the degree of \emph{Philosophiae Doctor} at the Université de Lyon.
The research presented here was conducted at the Laboratoire de Biométrie et Biologie Evolutive (LBBE), under the supervision of research director M. Nicolas Lartillot.
This work was conducted from September 2017 onward during a 3 years grant by ENS de Lyon (Contrat Doctoral Spécifique Normalien).
The thesis is a collection of three manuscripts preceded by an introduction that relates them together and provides background information and motivation for the work.