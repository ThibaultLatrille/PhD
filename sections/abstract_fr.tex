\thispagestyle{empty}

\section*{Résumé (v1)}

Dans cette thèse, plusieurs aspects de l'équilibre mutation-sélection-dérive sont étudiés et mis en relation avec des données empiriques, dans le contexte de séquences d'ADN codant pour des protéines.
Premièrement, parce que la composition des séquences d'ADN codant pour les protéines ne reflète pas le processus de mutation sous-jacent, mais son filtrage par la sélection au niveau des acides aminés, une modélisation phénoménologique minutieuse est nécessaire pour estimer le processus de mutation et le biais de fixation des nucléotides.
Deuxièmement, l'équilibre entre mutation et sélection est arbitré par la dérive, qui est médiée par la taille efficace de la population et ses changements le long d'une phylogénie peuvent être estimés par des modèles de codons mécanistes.
Enfin, la sélection pour la stabilité des protéines implique une relation analytique entre la pression de sélection et la taille efficace de la population et le niveau d'expression des protéines.

\section*{Résumé (v2)}

L'évolution moléculaire vise à caractériser les mécanismes à l'œuvre dans l'évolution des séquences, régie par un processus stochastique composé de mutation, sélection et de dérive génétique.
À long terme, ce processus stochastique résulte en une histoire d'événements de substitution le long des arbres d'espèces, induisant des motifs complexes de divergence moléculaire entre les espèces.
En analysant ces motifs, les modèles de codons phylogénétiques visent à estimer les paramètres d'évolution.
Dans ce contexte, cette thèse s'est concentrée sur les modèles de codons phylogénétiques et sur la modélisation de l'interaction entre la mutation, la sélection et la dérive des séquences d'ADN codant pour les protéines.
Parce que la composition des séquences d'ADN codant pour les protéines ne reflète pas le processus de mutation sous-jacent, mais qu'elle est filtrée par sélection au niveau des acides aminés, une modélisation minutieuse est nécessaire pour démêler la mutation et la sélection.
Par conséquent, j'ai d'abord développé un modèle d'inférence de codon phylogénétique dans lequel différents taux d'évolution donnent une représentation précise de la manière dont la mutation et la sélection s'opposent à l'équilibre.
Entre les forces opposées de mutation et de sélection, l'équilibre est arbitré par la dérive génétique, qui à son tour est modulée par la taille efficace de population ($\Ne$).
En conséquence, la variation de $\Ne$ le long d'une phylogénie peut théoriquement être déduite des motifs de substitutions le long des lignées.
J'ai ainsi développé un deuxième modèle d'inférence, reconstituant à la fois le paysage de fitness en chaque au site, les tendances à long terme de $\Ne$ et le taux de mutation le long de la phylogénie.
Ce cadre bayésien a été testé par rapport à des données simulées puis appliqué à des données empiriques.
Les estimations de la variation de $\Ne$ correspondent à la direction attendue de la corrélation avec les traits d’histoire de vie ou les variables écologiques, malgré que l'ampleur de la variation estimé soit plus étroite que prévue.
Afin de comprendre cette variation étroite du $\Ne$ estimé, j'ai finalement développé un modèle théorique décrivant comment les changements à la fois de $\Ne$ ou du niveau d'expression ou la protéine se traduisent par un changement du taux de substitution, sous l'hypothèse que les protéines mal repliés sont contre sélectionnées.
Cette réponse est paramétrée en termes de paramètres moléculaires de la biophysique des protéines et implique une faible réponse du taux de substitution aux changements de niveau d'expression ou de $\Ne$.


\section*{Résumé étendu}

La théorie neutre de l’évolution, et son extension quasi-neutre, ont profondément influencé notre compréhension de la génétique des populations et de l'évolution moléculaire.
Au-delà des disputes et des controverses entre neutralisme et sélectionnisme, le consensus actuel est de considérer l'évolution des séquences génétiques comme un processus stochastique.
Un composant de ce processus est la création de diversité par la mutation, tandis que la sélection filtre cette diversité.
Enfin l'équilibre entre ces composants est arbitré par la dérive génétique, elle même étant une fonction la taille efficace de population ($\Ne$).
Le résultat à long terme de ce processus évolutif stochastique est une accumulation de substitutions ponctuelles (à la fois synonymes et non synonymes) entre les espèces.
S'appuyant sur l'information contenue dans les alignements multiples de séquences d'ADN, l'objectif des modèles à codons phylogénétiques est de mieux caractériser et quantifier l'interaction entre mutation, sélection et dérive génétique.
Les modèles à codons sont toujours un domaine de recherche actif et se scindent en deux philosophies différentes.
D'un côté, les modèles phénoménologiques, visent à capturer l'effet net de la sélection à travers le paramètre $\dnds$, défini comme le ratio du taux substitutions non-synonymes (sous-sélection) sur le taux de substitution synonyme (supposé neutre).
De l'autre côté, des approches mécanistes ont pour objectif de modéliser le paysage du fitness.
En l'état, cependant, de nombreuses questions restent ouvertes et les modèles actuels, qu'ils soient phénoménologiques ou mécanistes, présentent de nombreuses faiblesses.
Les approches phénoménologiques pourraient encore être améliorées, tout en restant dans l'idée de ne pas modéliser explicitement le paysage détaillé du fitness.
Quant aux approches mécanistes, dans leurs versions actuelles, émettent des hypothèses très fortes, telles que l'indépendance entre sites, un paysage de fitness fixe, mais aussi une taille efficace de population ($\Ne$) constante le long de la phylogénie.
Plus fondamentalement, il existe un certain vide à combler entre ces deux approches, et de meilleures connexions pourraient être établies entre elles.

Dans ce contexte, mon travail de thèse représente une tentative de démêler les interactions complexes entre mutation, sélection et dérive génétiques à l'aide de modèles à codons phylogénétiques, selon les deux approches, phénoménologiques ou mécanistes.
Au cours de ce travail, j'ai testé des idées théoriques sur des données empiriques, en utilisant une combinaison de développements analytiques, d'expériences en simulation et d'inférence bayésienne.
Les résultats sont divisés en trois manuscrits indépendants.

Le premier article revient sur la question de l'équilibre entre biais de mutation et biais de sélection, et comment cet équilibre doit être correctement formalisé dans le contexte des modèles à codons classiques.
Parce que la composition des séquences d'ADN codant pour les protéines ne reflète pas le processus sous-jacent de mutation, mais son filtrage par sélection au niveau des acides aminés, une modélisation minutieuse est nécessaire pour démêler le processus de mutation et les biais nucléotidiques d'un côté, et la sélection d'un autre côté.
Malheureusement, les modèles à codons phénoménologiques modèles, développés à l'origine pour estimer le taux d'évolution sur les acides aminés ($\dnds$), ne modélisent pas correctement cet équilibre mutation-sélection.
En effet, ils utilisent le  biais de composition nucléotidique observé comme proxy pour le biais de mutation nucléotidique.
En conséquence, ils ne fournissent pas une estimation précise du processus de mutation, même s'ils sont capables d'estimer de manière assez fiable la force globale de sélection agissant sur les acides aminés.
Pour résoudre ce problème, j'ai développé un modèle à codon phylogénétique dans lequel la vitesse d'évolution n'est pas considérée comme un paramètre unique mais comme un tenseur (95 paramètres libres).
Le tenseur capture les faibles différences de taux de fixation (ou $\dnds$) dans différentes directions, ce qui donne une représentation précise de la manière dont la mutation et la sélection s'opposent à l'équilibre.
Cette paramétrisation est la forme paramétrique la plus simple, dans un contexte phénoménologique, capable de faire la séparation mutation-sélection de manière exacte (ou asymptotiquement exacte).
Grâce à cela, cette approche de modélisation donne une estimation fiable du processus de mutation, tout en démêlant les probabilités de fixation dans différentes directions.
Cela offre des outils pour mieux comprendre comment le processus mutation-mutation s’articule avec la conversion génique biaisée (gBGC), et permet donc également de mieux comprendre comment la gBGC affecte à la fois la composition nucléotidique et le $\dnds$.

Si premier manuscrit articule mutation et mutation dans le contexte d'un modèle à codon phénoménologique, l'équilibre entre ces forces de mutation et de sélection est arbitré par la dérive génétique, qui à son tour est modulée par taille efficace de population ($\Ne$).
En conséquence, théoriquement, la variation de $\Ne$ le long d'une phylogénie peut être déduite des traînées de substitutions le long des lignées.
Le deuxième manuscrit explore ainsi la question de la prise en compte de la variation à long terme de la taille efficace de population ($\Ne$) entre les espèces, dans le contexte d'un modèle de sélection de mutation mécaniste.
Les travaux présentés dans ce second manuscrit représentent la partie la plus intensive du travail de doctorat, en termes de modélisation, d'algorithmes de Monte-Carlo et de développement logiciel.
 J'ai ainsi développé un modèle de mutation-sélection mécaniste reconstituant le paysage le fitness en chaque site, les tendances à long terme de la taille efficace de population et du taux de mutation le long de la phylogénie, à partir d’alignements d'ADN de séquences codantes.
Simultanément, l'approche estime la corrélation entre les traits d’histoires de vie, le taux de mutation et la taille efficace de population, qui inclut explicitement l'inertie phylogénétique.
Ce modèle a été testé sur des données simulées, puis appliqué à des données empiriques chez les mammifères, les isopodes, les primates et les drosophiles.
Les résultats sur données simulées et empiriques suggèrent qu'il existe un des signaux persistants dans les séquences d’ADN qui permettent de reconstruire l'évolution de $\Ne$ le long de la phylogénie.
Par ailleurs les variations observées correspondent à la direction attendue de la corrélation avec les traits d’histoire de vie ou les variables écologiques.
Cependant, l'ampleur de la variation inférée de $\Ne$ à travers la phylogénie est plus étroite que prévu.

Finalement, ces observations de la gamme de variations de $\Ne$ relativement étroite mise à jour par cette approche entièrement mécaniste, m'ont amené à revoir la question de savoir comment la biophysique des protéines, et plus généralement l'épistasie, peut moduler quantitativement la réponse du processus évolutif moléculaire aux changements de la taille efficace de population.
Ce dernier travail est présenté comme un troisième manuscrit.
En effet ce second travail partait du principe que les lignées avec un grand $\Ne$ devraient subir une sélection purificatrice plus forte.
Cependant, les hypothèses sur la structure sous-jacente du paysage de la condition physique peuvent avoir une grande influence sur la réponse attendue de la probabilité moyenne de fixation aux changements de $\Ne$.
De plus, une augmentation du niveau d'expression des protéines entraîne également une diminution de la probabilité moyenne de fixation.
Cette corrélation est également prédite par des modèles théoriques supposant que les protéines mal repliées sont sous sélection négative.
En conséquence, il convient d'articuler ensemble toutes ces corrélations entre la probabilité moyenne de fixation, la taille efficace de population et le niveau d'expression, en rapport avec la structure du paysage de fitness sous-jacent.
Pour ce faire, j'ai dérivé une approximation théorique de la réponse de la probabilité moyenne de fixation aux changements à la fois de $\Ne$ et du niveau d'expression, en fonction du la relation génotype-phénotype-fitness sous-jacente.
Le développement est généralement valide pour des traits additifs sous une fonction de fitness log-concave, mais a été appliqué plus spécifiquement à un modèle biophysique dans lequel les protéines sont sous sélection directionnelle pour maximiser leur stabilité conformationnelle.
Dans ce cas précis, le modèle prédit une réponse faible de la probabilité moyenne de fixation aux changements de $\Ne$ ou de niveau d'expression (qui sont interchangeables), un résultat corroboré par des simulations sous des modèles plus complexes.
Sur la base de preuves empiriques, je propose que l'adéquation basée sur la stabilité conformationnelle pourrait ne pas fournir un mécanisme suffisant pour expliquer l'amplitude de la variation la pression de sélection qui est observée empiriquement.
D'autres aspects de la biophysique des protéines pourraient être explorés tels que les interactions protéine-protéine, qui pourraient conduire à une réponse plus forte pression de sélection aux changements de $\Ne$.
Cependant, il subsiste un écart entre les prévisions quantitatives des modèles biophysiques et les observations empiriques reliant la réponse de la pression de sélection aux changements de $\Ne$ et du niveau d'expression.

Pour conclure, ce travail est une tentative modeste de construire des modèles intégrés d'évolution des séquences d'ADN codant pour les protéines.
Il a réussi à consolider l'idée que les modèles de substitutions nous informent sur les fluctuations à long terme de la dérive le long des branches et la sélection le long des sites.
Il n'a pas réussi à modéliser le paysage du fitness, apparemment trop spécifique au site, ou intégré sur trop de sites.
C'est une indication que le paysage de fitness des séquences codant pour les protéines se situe entre ces deux extrêmes.
Ce travail représente aussi un cadre conceptuelle permettant de  relier phylogénie et génétique des populations.
Par exemple, confronter l'estimation du taux de substitution adaptative dans le cadre des modèles phylogénétiques d’un côté et dans le cadre génétique des populations d’un autre côté permet de construire un cadre unifié de l'évolution de l'ADN codant pour les protéines.
Enfin, je pense que cette thèse n'apporte aucun résultat révolutionnaire, ni perturbateur, mais consolide plutôt des modèles théoriques sur lesquels se fonde l'évolution moléculaire et souligne les écueils à éviter.
La science, tout comme un processus de sélection de mutation, n'est pas optimisée mais comme un compromis entre l'exploration de nouvelles idées et l'exploitation des anciennes.
Les tentatives de construction de ponts et de connexion entre les champs, tout comme la recombinaison, peuvent apporter le meilleur des deux mondes, mais de nombreuses tentatives sont nécessaires.
