\thispagestyle{empty}
\section*{Remerciements}


A celles et ceux qui ont parcouru avec moi un bout du chemin lors de cette expédition, vous m’avez fait découvrir des lieux d’émerveillements, autant hors des sentiers battus que sur les lieux pittoresques, vous m’avez sorti des impasses et guidé à travers les ravines et les combes.

Un immense merci à$\hdots$ \textit{I’m immensely thankful to$\hdots$}

$\hdots$ bien évidemment, celui qui a dessiné la carte, et sillonné ce périple à mes côtés, Nicolas Lartillot.
Tu m’as guidé tout au long cette traversée, ta vision d’ensemble ainsi que ta précision dans les détails techniques nous ont amenés sur des pics et des vallées que je n’aurai jamais imaginées exister.
Tes discussions et échanges autant scientifiques qu’humains ont sillonné d'innombrables sujets, nourri par ton indéniable sens aiguisé de l’observation et tes incroyables qualités d’empathie.
J’espère qu’on aura l’occasion de refaire une ou plusieurs randonnées ensemble.

$\hdots$ \textit{my jury who accepted reviewing this thesis, Céline Brochier-Armanet, Julien Yann Dutheil, Richard Goldstein, Carina Farah Mugal.
These pages are a map of a three years’ scientific journey spent in Lyon, I hope you’ll be enjoying the views of fitness landscapes, adaptive peaks and Markov chains that Nicolas and I visited along the way.}

$\hdots$ vous qui peupliez la vallée du LBBE, autant les habitants endémiques que les individus migrateurs campant le temps d’un passage.
Vous avez construit et entretenu une communauté curieuse, incroyablement compétente et bienveillante, merci.

$\hdots$ celles et ceux que j’ai croisés le long de votre aventure et qui se sont envolés vers une autre vallée, Aline Muyle, Wandrille Duchemin, Adrián Arellano Davín, Héloïse Phillippon, Pierre Garcia, Monique Aouad, Anne Oudard, Frédéric Jauffrit, Maud Gautier, Samuel Barreto, Vincent Lanore, François Gindraud et Diego Hartasánchez Frenk.
Félicitations, Aline, pour ton retour !

$\hdots$ vous qui profitez encore du paysage et de la sagesse des résidents du LBBE encore quelque temps, Florian Bénitière, Alexandre Laverré, Alexia Nguyen Trung, Djivan Prentout, Théo Tricou, Marina Brasó Vives, Claire Gayral, Hugo Menet, Louis Duchemin, Antoine Villié, Alice Genestier et Julien Joseph.
Je n’oublierai point que nous avons ensemble parcouru les sinuosité des Dombes, des gorges, des canyons, des lacs et même des grottes, autant au sens littéral que figuré.

$\hdots$ celles et ceux qui ont construit le refuge, Anamaria Necsulea, Damien de Vienne, Dominique Mouchiroud, Hélène Badouin, Laurent Duret, et tant d’autres.
Ainsi que vous qui avez semé les poireaux et les carottes, Annabelle Haudry, Bastien Boussau, Éric Tannier et Vincent Daubin.

$\hdots$ celles qui m’ont guidé à travers la jungle et les ronces administratives, Nathalie Arbasetti, Odile Mulet-Marquis, Laetitia Catouaria et Aurélie Zerfass, j’en ressors sans trop d'égratignures grâce à vous, alors que nous savons pertinents que j’y aurai laissé des plumes si vous n’étiez pas là pour fournir votre aide si précieuse.

$\hdots$ ceux qui m’ont fourni les outils et le matériel de randonnée et d’escalade, et en plus m’ont appris à m’en servir en toute sécurité, Bruno Spataro, Stéphane Delmotte, Simon Penel, Adil El Filali, Vincent Miele et Philippe Veber.

$\hdots$ celles et ceux avec qui j’ai participé aux formations des futurs grimpeuses et grimpeurs, Marie Sémon, Carine Rey, Corentin Dechaut, Vincent Lacroix, Arnaud Mary, et tant d’autres déjà cités plus haut.

$\hdots$ à Diego pour tes remarques et commentaires aiguisés, et encore merci pour avoir relu et corrigé d’innombrables fautes.

$\hdots$ au Ministère de l’Enseignement Supérieur et de la Recherche, et en réalité à la société pour avoir sponsorisé cette expédition, et financer les victuailles, le matériel, le transport, et le logement.

$\hdots$ à toute la tribu d'amies et d'amis pour votre confiance et tous ces moments chaleureux que j'ai passé à vos cotés.
Autant celles et ceux qui ont déjà soutenu que ceux qui y aspirent, en passant par ceux qui en sont curieux et interessé, sans oublier ceux qui me demandent pourquoi je fais ça.
J'espère vous donnez l'envie d'explorer quelques pages de ce paysage, je vous promet qu'au travers de cette marche (aléatoire) il y a vraiment des pics (adaptifs), des vallées (de fitness), des chaines (de Markov), des barrières (de dérive), et tant d'autres curiosités géologiques.

$\hdots$ à toute la famille, maman et papa qui m’ont rendu diploïde, et m’ont appris à lire une carte, sans -trop- se perdre et aussi pour m’avoir appris à se retrouver si jamais on était perdu.
Iris et Myriam, qui ont élu dans domicile et découvert d'éblouissants et incroyables recoins de paradis, vous avez bien fait de suivre votre instinct et ne pas avoir suivi la route toute tracée qui ne vous aurez certainement pas amené dans cette ecosystéme si chaleureux et accueillant.
Samuel, qui m’a ravitaillé en patate tout le long de la dernière montée depuis son refuge branché à la fibre.

$\hdots$ Iris pour cette magnifique aquerelle en préface, ce fut un véritable emmerveillement de voir se réaliser sous ton pinceau la représentation figuré que je me suis fais de la thèse.

$\hdots$ à Judith, qui a partagé l’ensemble de cette aventure avec moi, sur les plus somptueux pics et les gorges le plus escarpées.
Après ce col, je ne sais pas dans quelle vallée nous irons voyager et randonnée ensemble, mais avec toi je n’ai pas peur d’y aller même si nous savons pertinemment que Lyon et sa tribu chaleureuse nous manqueront.

\begin{flushright}
	Thibault Latrille,
	
	Villeurbanne
\end{flushright}
